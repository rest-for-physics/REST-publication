\section{REST conceptual design and scope}
\label{sec:conception}

REST-for-Physics defines common data structures for event-based data processing. As we will see later on the general description (at section\,\ref{sec:framework}) this entails a prototyping of the event data holder, the processes that transform or operate those data holders, and the description of the metadata information giving a meaning to the data being processed; such as initial data taking conditions, input processing parameters, or output results written to disk in the form of metadata.
The prototypes to event data, processes and metadata are complemented  with basic analysis tools that are frequently used on event-based data analysis. Another important structure, named tree, is used to gather relevant event information during the data processing. This analysis summary tree contains a set of variables defined during the event data processing to be used in subsequent higher level analysis.

The intention of REST-for-Physics is to define a framework, or code development space, that centralizes related event processing and analysis routines, being those routines contributed by the same experts that work on the analysis of the data. The REST community keeps a strong link between the algorithm design and the framework design, since there is an implicit connection between the algorithm development, analysis interpretation, and framework design requirements. REST provides already existing processes that we will be able to use directly to define a given event processing task. Anyhow, REST has been designed to provide the means to be extended with new processes, metadata or event data types.

The development of REST emerges in a strong academical environment. In such context, REST intents to provide the means for academic works to materialize in the form of a piece of code that can be re-used within an already consolidated software infrastructure. A major goal for REST is to make it more accessible to non-computing experts that have a high level for algorithm coding abstraction and comprehension of the physics context.

REST-for-Physics does not replace or compete with other dedicated simulation packages which provide high accuracy physics description on dedicated problems, but it seeks to integrate those packages, such as Geant4~\cite{Agostinelli:2002hh} or Garfield++~\cite{Garfield}, and exploit them inside the framework as needed on the processing of the event data. In addition, REST-for-Physics includes dedicated libraries (described at section\,\ref{sec:libraries}) that implement specialized algorithms for signal processing or physical track reconstruction. We develop our own algorithms for known mathematical problems, e.g. time signal processing, to have full control over those  and adapt them to our experimental needs, while still linking to consolidated libraries when possible, as it is for example the case for high-precision numbers implementation at the \emph{mpfr} library\,\cite{10.1145/1236463.1236468} or the use of graph theory methods\,\cite{Applegate:2007:TSP:1374811,concorde}.