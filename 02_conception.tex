\section{REST conceptual design and scope}
\label{sec:conception}

% What it is?
%REST-for-Physics (Rare Event Searches Toolkit for Physics) is a collaborative software effort providing a common framework and tools for acquisition, simulation, generic data analysis, and detector response in experimental particle physics. An ambitious feature of REST-for-Physics is its capability to analyze together and compare Monte Carlo and experimental data using the same \emph{event processing} routines upon a unified \emph{event data} - \emph{metadata} architecture. 

% Describe the main elements of REST. It can be better seen as extending the scope of ROOT development framework to our domain.

%% Give a first idea on the fact that it focuses on detailed/accurate description and analysis on an event-per-event based data processing flow.
REST-for-Physics is a software development framework aiming to define common data structures for event-based data processing. As we will see later on the general description (at section\,\ref{sec:framework}) this entails a prototyping of, the event data holder, the processes that transform or operate those data holders, and the description of the metadata information giving a meaning to the data being processed; such as initial data taking conditions, input processing parameters, or output results written to disk in the form of metadata.
The prototypes to event data, processes and metadata are complemented  with basic analysis tools that are frequently used on event-based data analysis. Another important structure, named tree, is used to gather relevant event information during the data processing. This analysis summary tree contains a set of variables defined during the event data processing that will be used for a more sophisticated and dedicated analysis in the future.

The philosophy of REST-for-Physics is to define a framework, or code development space, that centralizes related event processing and analysis routines, being those routines contributed by the same experts that work on the analysis of the data. The REST community keeps a strong link between the algorithm design and the framework design, since there is an implicit connection between the algorithm development, analysis interpretation, and framework design requirements. REST provides already existing processes that we will be able to use directly to define a given event processing task. Anyhow, REST has been designed to provide the means to be extended with new processes, metadata or event data types.

The development of REST emerges in a strong academical environment. In such context, REST pretends to provide the means for academic works to materialize in the form of a piece of code that can be re-used within an already consolidated software infrastructure. A major goal for REST is to make it more accessible to non-computing experts that have a high level for algorithm coding abstraction and comprehension of the physics context.

REST-for-Physics does not replace or compete with other dedicated simulation packages which provide high accuracy physics description on dedicated problems, but it seeks to integrate those packages, such as Geant4~\cite{Agostinelli:2002hh} or Garfield++~\cite{Garfield}, and exploit them inside the framework as needed on the processing of the event data. In addition, REST-for-Physics includes dedicated libraries (described at section\,\ref{sec:libraries}) that implement specialized algorithms for signal processing or physical track reconstruction. There might be available dedicated libraries for that purpose on the market, however, in our domain is an advantage to have full control over those algorithms, while still linking to consolidated libraries when possible, as it is for example the case for high-precision numbers with the \emph{mpfr} library\,\cite{10.1145/1236463.1236468} or the use of graph theory methods\,\cite{Applegate:2007:TSP:1374811,concorde}.

%% Differentiation with other package-code projects on the "market". Say that it emerges in a strong academical environment. bla bla

%% Therefore, the REST community keeps a strong link between the algorithm development and the framework design. Since there is a kind of connection between the needs of algorithm development, analysis interpretation, and framework design needs. Completely open-source (publicly exposed) code available.

%% REST includes also includes algorithms - signal-processing bla bla bla - unprecent try to centralize those algorithms used ... in a single platform.

%% There might be other dedicated and sophisticated libraries for signal processing, or graph theories, etc etc. It is not our intention to replace those, and when possible, we encourage the connection to external libraries for (concorde, mpfr, etc). Say we need to design the algorithms to be part of global infraestructure.

