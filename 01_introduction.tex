\section{Introduction}
\label{sec:intro}

% What it is?
REST-for-Physics (Rare Event Searches Toolkit for Physics) is a collaborative software effort providing a common framework and tools for acquisition, simulation, generic data analysis, and detector response in experimental particle physics. An ambitious feature of REST-for-Physics is its capability to analyze together and compare Monte Carlo and experimental data using the same \emph{event processing} routines upon a unified \emph{event data} - \emph{metadata} architecture. 

% Describe motivation. 
The framework was born to bring together different software requirements related to gaseous Time Projection Chambers (TPCs) in the context of Rare Event Searches, and to unify and coordinate various independent developments in a common, modular infrastructure with potential for scalability and reusability. Special care has been taken to ensure the traceability and reproducibility of the results obtained after the data processing, linking the code version with the metadata version stored on disk, and protecting such relation. Any user local changes to the code are identified at compilation time. This is used to guarantee that the version executed and the results written to disk correspond to an unmodified official public release. This fact is extremely relevant when planning to register official experimental data and preserve it for historical reasons, such as covering the data management plan of scientific collaborations, including the release  of data to be publicly exploited outside the collaboration domain. The code updates are periodically published at the Zenodo citations system, where one finds the latest official release today\,\cite{javier_galan_2021_5092550}.

% Add historical context
REST-for-Physics is the result of several years of experience on detector physics and research, motivated originally to cover the software needs of the T-REX project for neutrino and dark matter searches~\cite{Irastorza:2015dcb,Irastorza:2015geo}. The REST-for-Physics code has benefited from several academic works, as it becomes apparent in several PhD thesis publications\,\cite{IguazThesis,tomas2013development,SeguiThesis,HerreraThesis,GraciaThesis, GarciaPascualThesis, RuizThesis} that have contributed to shape and define the final project that we describe in this manuscript.
This project has contributed to the development of different but interconnected research activities in a coherent way, unifying common tools that are used today regularly not only in research but also at all the academic levels, from undergraduate to master students. 

% REST-in-a-nutshell

% Research and experiments, distribution.
Different experimental projects have seen REST-for-Physics growing from its preliminary stages to the mature project we present in this work. REST-for-Physics has been evolving within, and it is being used to produce results at, CAST~\cite{Anastassopoulos:2017ftl}, TREX-DM~\cite{trexdm_bckmodel}, PandaX-III~\cite{pandaxiii_cdr,Lin:2018mpd,Galan:2019ake}, and IAXO~\cite{Armengaud:2019uso}. Those projects have benefited from the consolidation of REST as a common tool widely used among collaborators to process, register and analyze detector data. We foster the use and development of REST in other experiments in a community effort to maintain appropriate tools for related tasks. In addition to sharing the know-how and experience in our physics domain, the motivation to release a public common framework resides in providing the possibility to distribute the experimental data following a unique format readable with REST-for-Physics, or any other ROOT I/O compatible code in a future open-data program of the experiments. The code is open-source and it is distributed under a GNU public license at \emph{GitHub}\,\cite{REST_Git}.

% Introducing scope of this paper
The aim of this document is to give the reader a broad perspective of the purpose of the software project, its organization and contents, and the basic instruments that shape the whole infrastructure, giving an idea of its scalability potential, and in addition, showing the code validation strategy and continuous integration philosophy. For further reference we refer to detailed information, including an API class documentation for developers\,\cite{REST_API}  synchronized daily with the latest development version, and a comprehensive guide for first time users\,\cite{REST_user_guide}. We also offer an additional communication channel in the form of a public forum\,\cite{REST_forum} to encourage discussion about topics related in our domain, help others on their first steps using REST and/or integrate their first routines inside the framework, and discuss about new or existing feature upgrades.