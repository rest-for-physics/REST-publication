\section{Introduction}
\label{sec:intro}

% Motivation for software development

% Emphasize also its academic use

% What it is?
REST-for-Physics (Rare Event Searches Toolkit for Physics) is a collaborative software effort providing a common framework and tools for acquisition, simulation, generic data analysis, and detector response in experimental particle physics. An ambitious feature of REST-for-Physics is its capability to analyze together and compare Monte Carlo and experimental data using the same \emph{event processing} routines upon a unified \emph{event data} - \emph{metadata} architecture. 

% Describe motivation. 
The framework was born to bring together different software requirements related to gaseous Time Projection Chambers (TPCs) in the context of Rare Event Searches, and to unify and coordinate various independent developments in a common modular infrastructure with potential for scalability and reusability. Special care has been taken to assure the traceability and reproducibility of the results obtained after the data processing, linking the code version with the metadata version stored on disk, and protecting such relation. Any user local changes are identified at compilation time, and it is used to assure that the version executed and the results written to disk correspond to an unmodified official public release. This fact is extremely relevant when planning to register official experimental data and preserve it for historical reasons. Our code updates are periodically published at the zenodo citations system, where we find our latest official release today\,\cite{javier_galan_2021_5092550}.

% Add historical context?
REST-for-Physics is the result of several years experience on detector physics and research, motivated originally to cover the software needs at the T-REX project for neutrino and dark matter searches~\cite{Irastorza:2015dcb,Irastorza:2015geo}. The REST-for-Physics code has benefited from several academic works\,\cite{IguazThesis,tomas2013development,SeguiThesis,HerreraThesis,GraciaThesis, GarciaPascualThesis, RuizThesis} that have contributed to shape and define the final project that we describe in this publication.
This project has contributed to develop different but interconnected research activities in a coherent way, unifying common tools that are used today regularly not only in research, including the aforementioned PhD. thesis, but also at all the academic levels, from undergraduate to master students. 

Different experimental projects have seen REST-for-Physics growing from its preliminary stages to the mature project we present in this work. REST-for-Physics has been evolved and it is being used to produce results at CAST~\cite{Anastassopoulos:2017ftl}, TREX-DM~\cite{trexdm_bckmodel}, PandaX-III~\cite{pandaxiii_cdr,Lin:2018mpd,Galan:2019ake}, and IAXO~\cite{Armengaud:2019uso}. In addition to sharing the know-how and experience in our physics domain, the motivation to release a public common framework resides on providing the possibility to distribute the experimental data following a unique format readable with REST-for-Physics, or any other \emph{ROOT} I/O compatible code in a future open-data program of the experiment. The code is open-source and it is distributed under a GNU public licence at \emph{GitHub}\,\cite{REST_Git}.

The aim of this document is to give the reader a broad perspective of the purpose of the software project, its organization and contents, and the basic instruments that shape the whole infrastructure, giving an idea of its scalability potential, and in addition, showing the code validation strategy and continuous integration philosophy. For further reference we provide detailed information, including an API class documentation for developers\,\cite{REST_API} daily synchronized with the latest development version, and a comprehensive guide for first time users\,\cite{REST_user_guide}. We also offer an additional communication channel in the shape of forum\,\cite{REST_forum} to encourage discussion about topics related in our domain, help others on their first steps using REST, or integrate their first routines inside the framework, and discuss about new or existing features upgrades.

% The class code documentation is compiled using \emph{Doxygen} markdown and hosted at a server at the University of Zaragoza\footnote{\url{https://sultan.unizar.es/rest}}. There is a documentation user guide available at XXX.

% Projects involved in the use and development of REST: CAST, IAXO, PandaX-III, TREX-DM.

%% Monte Carlo and experimental common processing discussion
%% This is possible by using those \emph{event processes} to condition the input data generated, for example, by a \emph{Geant4} Monte Carlo simulation. After an appropriate \emph{event data} conditioning, our Monte Carlo generated event will reproduce the \emph{rawdata} of the detector acquisition. Once we are at that stage, we can benefit from using the same \emph{event processes} to analyze Monte Carlo and real experimental data. A realistic Monte Carlo \emph{rawdata} reconstruction will allow us to assess, validate and optimize the processes that will be involved in the real event reconstruction and analysis even before the start of the physics run of the experiment.


% Results already produced by REST in the past

% Academic impact of REST-for-Physics