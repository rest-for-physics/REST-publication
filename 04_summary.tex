\section{Summary}


%We are not presenting new technology or advanced bla bla bla, we are putting together the tools we use to process and analyse data in our physics domain context. Giving it robustness, bla bla bla

%We use ROOT to avoid worrying about I/O serialization and to find a solution bla bla bla, large community, long term support, etc.

%Future RDataFrame

We present in this work an overview of the REST-for-Physics (Rare Event Searches Toolkit for Physics) ecosystem, providing a broad perspective to the infrastructure and organization of the project as a whole. The REST-for-Physics framework is a ROOT-based solution providing the means to process and analyse experimental or Monte Carlo event data. The framework development has been motivated to cover the needs at Rare Event Searches experiments, and its components naturally implement tools to address the challenges in this kind of experiments; such as the integration of a detector physics response, the implementation of signal processing routines, and topological algorithms for physical event identification. In spite of this specialization, the framework was conceived for scalability, and other event-oriented applications could benefit from the data processing and/or metadata description implemented in REST, being the generic framework tools completely decoupled from dedicated libraries.

REST-for-Physics is a consolidated piece of software already serving the needs of different experiments, as well as related research. As such, special care has been taken on the traceability of the code and the validation of the results produced within the framework, and the connectivity between code and data registered through specific version metadata members.