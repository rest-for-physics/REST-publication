\section{Summary}

In this work we have given a broad overview of the REST-for-Physics framework and its different components. Our aim was to provide the reader with a general idea of the philosophy, structure and organization of the software project. And, without entering into great detail, provide an overview of the present use and functionality of REST-for-Physics.

The REST-for-Physics framework and libraries are a natural extension of ROOT, since the most basic elements inherit directly from TObject. We utilize the ROOT I/O serialization to manage the data storage while focusing on the development of physical processes that provide to REST its functionality. The motivation for this choice is the experience we acquired on the ROOT framework, and the benefit of using the analysis tools it provides. ROOT was born already more than 25\,years ago and it is still strongly supported and actively maintained by the CERN community which counts with thousands of users. ROOT is exhaustively used in particle physics today, and its continuity in the long term seems to be guaranteed by CERN.

The REST-for-Physics framework fully exploits the schema evolution from ROOT in order to minimize the impact on data member changes in \emph{specific event} or \emph{metadata} objects, thus making files written with REST to be backward- and forward-compatible. One of the key aspects of the REST-for-Physics code, crucial for the storage and processing of experimental data, is its versioning strategy that it was carefully described in this paper. Such versioning strategy provides a unique relation between the code and the registered data, ensuring data and code traceability, leading to reproducible results.

One of the main motivations of the development of REST-for-Physics is to  collect and centralize the software efforts and progress on detector physics for the construction of low-background detection technologies. As such, REST-for-Physics aims to serve as a platform to support future contributions in the field, consolidating common processing routines on event reconstruction, signal conditioning or pattern recognition. REST has been widely tested using gaseous TPCs, although its routines share many aspects with other detector technologies: some of the routines could be directly exploited by other technologies, while others would require minor changes to be useful for other detection setups.
